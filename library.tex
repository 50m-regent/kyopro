\documentclass{jsarticle}
\usepackage{listings}
\lstset{
  basicstyle={\ttfamily},
  identifierstyle={\small},
  commentstyle={\smallitshape},
  keywordstyle={\small\bfseries},
  ndkeywordstyle={\small},
  stringstyle={\small\ttfamily},
  frame={tb},
  breaklines=true,
  columns=[l]{fullflexible},
  numbers=left,
  xrightmargin=0zw,
  xleftmargin=3zw,
  numberstyle={\scriptsize},
  stepnumber=1,
  numbersep=1zw,
  lineskip=-0.5ex
}
\renewcommand{\lstlistingname}{ソースコード}
\begin{document}
\section*{りーぜんと}
\subsection*{テンプレ}
\begin{lstlisting}[caption=テンプレ]
#include<bits/stdc++.h>
using namespace std;
#define int long long
const int INF = 1e18;
const int MOD = 1e9 + 7;
\end{lstlisting}
\begin{lstlisting}[caption=ループ]
#define _overload(_1, _2, _3, name, ...) name
#define _rep(i, n) repi(i, 0, n)
#define repi(i, a, b) for(int i = (int)(a); i < (int)(b); i++)
#define rep(...) _overload(__VA_ARGS__, repi, _rep)(__VA_ARGS__)
#define _rev(i, n) revi(i, n, 0)
#define revi(i, a, b) for(int i = (int)(a - 1); i >= (int)(b); i--)
#define rev(...) _overload(__VA_ARGS__, revi, _rev)(__VA_ARGS__)
#define each(i, n) for(auto&& i: n)
\end{lstlisting}

\subsection*{データ構造}
\begin{lstlisting}[caption=UnionFind]
struct UnionFind {
    vector<int> t;
    UnionFind(int size): t(size, -1) {}
    int root(int x) {
        return t[x] < 0 ? x : t[x] = root(t[x]);
    }
    int size(int x) {
        return -t[root(x)];
    }
    bool isSame(int x, int y) {
        return root(x) == root(y);
    }
    bool unite(int x, int y) {
        x = root(x), y = root(y);
        if(x != y) {
            if(t[y] < t[x]) swap(x, y);
            t[x] += t[y], t[y] = x;
        }
        return x != y;
    }
};
\end{lstlisting}
\begin{lstlisting}[caption=セグ木]
struct SegmentTree {
  int n = 1;
  vector<int> t;
  SegmentTree(vector<int> v) {
      while(n < v.size()) n *= 2;
      t.resize(2 * n - 1, INF);
      rep(i, v.size()) t[i + n - 1] = v[i];
      rev(i, n - 1) t[i] = min(t[2 * i + 1], t[2 * i + 2]); // 区間最小
  }
  void update(int x, int val) {
      x += n - 1;
      t[x] = val;
      while(x > 0){
          x--; x /= 2;
          t[x] = min(t[2 * x + 1], t[2 * x + 2]); // 区間最小
      }
  }
  int query(int a, int b, int now = 0, int l = 0, int r = -1) {
      if(r < 0) r = n;
      if(r <= a || b <= l) return INF; // minの単位元
      if(a <= l && r <= b) return t[now];
      int c1 = query(a, b, 2 * now + 1, l, (l + r) / 2);
      int c2 = query(a, b, 2 * now + 2, (l + r) / 2, r);
      return min(c1, c2);
  }
};
\end{lstlisting}
\begin{lstlisting}[caption=回文]
// 最長回文 v[i]がi文字目中心の差最大値
vector<int> manacher(string &s) {
    vector<int> radius(s.size());
    int i = 0, j = 0;
    while(i < s.size()) {
        while(i - j >= 0 && i + j < s.size() && s[i - j] == s[i + j]) {
            ++j;
        }
        radius[i] = j;
        int k = 1;
        while(i - k >= 0 && i + k < s.size() && k + radius[i - k] < j) {
            radius[i + k] = radius[i - k];
            ++k;
        }
        i += k;
        j -= k;
    }
    return radius;
}
\end{lstlisting}

\subsection*{数学}
\begin{lstlisting}[caption=素因数分解]
map<int, int> factorization(int n) {
    map<int, int> res;
    rep(i, 2, sqrt(n) + 1) {
        if(i > 3) i++;
        while(!(n%i)) {
            res[i]++;
            n /= i;
        }
    }
    if(n > 1) {
        res[n]++;
    }
    return res;
}
\end{lstlisting}
\begin{lstlisting}[caption=約数列挙]
vector<int> divisor(int n) {
    vector<int> res;
    rep(i, 1, sqrt(n) + 1) {
        if(!(n % i)) {
            res.push_back(i);
            if(i * i < n) res.push_back(n / i);
        }
    }
    return res;
}
\end{lstlisting}
\begin{lstlisting}[caption=最小公倍数]
int lcm(int a, int b) {return a / gcd(a, b) * b;}
\end{lstlisting}
\begin{lstlisting}[caption=累乗]
int factorial(int a) {return a < 2 ? 1 : factorial(a - 1) * a;}
\end{lstlisting}
\begin{lstlisting}[caption=総和]
int summation(int a) {return a < 1 ? 0 : (a * a + a) / 2;}
\end{lstlisting}
\begin{lstlisting}[caption=組み合わせ]
int comb(int n, int r) {
  int res = 1;
  rep(i, 1, r + 1) {
    res *= n--;
    res /= i;
  }
  return res;
}
\end{lstlisting}
\begin{lstlisting}[caption=素数判定]
bool isPrime(int n) {
  rep(i, 2, sqrt(n) + 1) {
    if(i > 3) i++;
    if(!(n % i)) return false;
  }
  return true;
}
\end{lstlisting}

\subsection*{bound}
\begin{lstlisting}[caption=bound]
template<class T>
int lb(vector<T>& x,T n){return lower_bound(all(x) , n) - x.begin();}
template<class T>
int ub(vector<T>& x,T n){return upper_bound(all(x) , n) - x.begin();}
\end{lstlisting}

\subsection*{MOD}
\begin{lstlisting}[caption=二分累乗とか]
int extgcd(int a, int b, int &x, int &y) {
  int g = a;
  x = 1, y = 0;
  if(b) {
    g = extgcd(b, a % b, y, x);
    y -= a / b * x;
  }
  return g;
}
  
int invmod(int a, int m = MOD) {
  int x = 0, y = 0;
  extgcd(a, m, x, y);
  return (x + m) % m;
}

int modpow(int a, int n){
  if(n < 1) return 1;
  return modpow(a * a % MOD, n / 2) * ((n % 2) ? a : 1) % MOD;
}

int modfact(int a){
  if(a < 2) return 1;
  return a * modfact(a - 1) % MOD;
}
\end{lstlisting}

\section*{やの}
\subsection*{テンプレ}
\begin{lstlisting}[caption=テンプレ]
#include<iostream>
#include<vector>
#include<algorithm>
#include<string>
#include<map>
#include<set>
#include<stack>
#include<queue>
#include<math.h>
using namespace std;
#define int long long
#define INF 1000000000
#define LLINF 9223372036854775807
#define mod 1000000007
\end{lstlisting}
\begin{lstlisting}[caption=alias]
typedef vector<int> VI;
typedef pair<int, int> pii;
typedef priority_queue<int> PQ;
#define SORT(c) sort((c).begin(),(c).end())
#define rSORT(c) sort((c).rbegin(),(c).rend())
\end{lstlisting}
\begin{lstlisting}[caption=ループ]
#define fore(i,a) for(auto &i:a)
#define REP(i,n) for(int i=0;i<n;i++)
#define eREP(i,n) for(int i=0;i<=n;i++)
#define FOR(i,a,b) for(int i=(a);i<(b);++i)
#define eFOR(i,a,b) for(int i=(a);i<=(b);++i)
\end{lstlisting}

\subsection*{データ構造}
\begin{lstlisting}[caption=UnionFind]
class UnionFind {
  public:
    vector<int> Parent;   
    UnionFind(int N) {
      Parent = vector<int>(N, -1);
    }
    int root(int A) {
      if (Parent[A] < 0) return A;
      return Parent[A] = root(Parent[A]);
    }
    int size(int A) {return -Parent[root(A)];}
     
    bool connect(int A, int B) {
      A = root(A);
      B = root(B);
      if (A == B) return false;
     
      if (size(A) < size(B)) swap(A, B);
     
      Parent[A] += Parent[B];
      Parent[B] = A;
     
      return true;
    }
};
\end{lstlisting}
\begin{lstlisting}[caption=セグ木]
struct segtree {
  public:
    const int SIZE = 1 << 18;
    VI seg, lazy;
    misegtree() :seg(SIZE * 2, INF), lazy(SIZE * 2, INF) {}
    void lazy_evaluate(int k, int l, int r) {
      if (lazy[k] != 0) {
        //seg[k] += lazy[k];
        seg[k] = min(seg[k], lazy[k]);
        if (r - l > 1) {
          //lazy[k * 2 + 1] += lazy[k];
          //lazy[k * 2 + 2] += lazy[k];
          //lazy[k * 2 + 1] = min(lazy[k * 2 + 1], lazy[k]);
          //lazy[k * 2 + 2] = min(lazy[k * 2 + 2], lazy[k]);
          //lazy[k * 2 + 1] = max(lazy[k * 2 + 1], lazy[k]);
          //lazy[k * 2 + 2] = max(lazy[k * 2 + 2], lazy[k]);
        }
        lazy[k] = LLINF;
      }
    }
    void update(int a, int b, int k, int l, int r, int x) {
      lazy_evaluate(k, l, r);
      if (r <= a || b <= l)return;
      if (a <= l && r <= b) {
        //lazy[k] += x;
        //lazy[k] = min(lazy[k], x);
        //lazy[k] = max(lazy[k], x);
        lazy_evaluate(k, l, r);
      } else {
        update(a, b, k * 2 + 1, l, (l + r) / 2, x);
        update(a, b, k * 2 + 2, (l + r) / 2, r, x);
        //seg[k] = seg[k * 2 + 1] + seg[k * 2 + 2];
        //seg[k] = min(seg[k * 2 + 1], seg[k * 2 + 2]);
        //seg[k] = max(seg[k * 2 + 1], seg[k * 2 + 2]);
      }
    }
    int query(int a, int b, int k, int l, int r) {
      lazy_evaluate(k, l, r);
      if (r <= a || b <= l)return LLINF;
      if (a <= l && r <= b)return seg[k];
      int x = query(a, b, k * 2 + 1, l, (l + r) / 2);
      int y = query(a, b, k * 2 + 2, (l + r) / 2, r);
      //return x + y;
      //return min(x, y);
      //return max(x, y);
    }
    void update(int a, int b, int x) { update(a, b, 0, 0, SIZE, x); }
    int query(int a, int b) {
      return query(a, b, 0, 0, SIZE);
    }
};
\end{lstlisting}

\subsection*{グラフ}
\begin{lstlisting}[caption=ダイクストラ]
const int MAX_N = 100;
int G[100][100];
int N;
VI dist(110);
  
void dikstra(int s) {
  priority_queue<pii, vector<pii>, greater<pii> >que;
  dist[s] = 0;
  que.push(pii(0, s));
  while (!que.empty()) {
    pii p = que.top();
    que.pop();
    int v = p.second;
    if (dist[v] < p.first) continue;
    REP(i, N) {
      if (G[v][i] == LLINF) continue;
      if (dist[i] > dist[v] + G[v][i]) {
        dist[i] = dist[v] + G[v][i];
        que.push(pii(dist[v], i));
      }
    }
  }
}
\end{lstlisting}
\begin{lstlisting}[caption=強結合なんとか]
const int MAX_N=10000;
vector<VI> g1(MAX_N);
vector<VI> g2(MAX_N);
VI flag(MAX_N);
int fcnt = 0;
int gcnt = 0;
vector<bool>vis(MAX_N, false);
VI group(MAX_N);
  
void dfs(int v) {
  vis[v] = true;
  for (int w : g1[v]) {
    if (!vis[w])dfs(w);
  }
  flag[fcnt] = v;
  //cout << fcnt << " " << v << endl;
  fcnt++;
}
  
void rdfs(int v) {
  vis[v] = false;
  group[v] = gcnt;
  //cout << v << " " << gcnt << endl;
  for (int w : g2[v]) {
    if (vis[w])rdfs(w);
  }
}
\end{lstlisting}

\subsection*{bound}
\begin{lstlisting}[caption=bound]
#define LB(x,a) lower_bound((x).begin(),(x).end(),(a))
#define UB(x,a) upper_bound((x).begin(),(x).end(),(a))
\end{lstlisting}

\subsection*{MOD}
\begin{lstlisting}[caption=MOD逆元]
int modpow(int a, int p) {
  if (p == 0) return 1;
  if (p % 2 == 0) {
    int halfP = p / 2;
    int half = modpow(a, halfP);
   
    return half * half % mod;
  } else {
    return a * modpow(a, p - 1) % mod;
  }
}
   
int comb(int a, int b) {
  if (b > a - b) return comb(a, a - b);
  int ansMul = 1;
  int ansDiv = 1;
  for (int i = 0; i < b; i++) {
    ansMul *= (a - i);
    ansDiv *= (i + 1);
    ansMul %= mod;
    ansDiv %= mod;
  }
    
  int ans = ansMul * modpow(ansDiv, mod - 2) % mod;
  return ans;
}
\end{lstlisting}

\end{document}